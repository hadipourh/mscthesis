\chapter*{چکیده}
\baselineskip=1cm
در این پایان‌نامه، بر پایه‌ی 
\cite{kreuzer2009algebraic}،
گردایه‌ای از روش‌های جبری در رمزشکنی را ارائه می‌دهیم. بعد از معرفی مقدمات رمزنگاری و مفاهیم اولیه‌ی حمله‌های جبری و بررسی چندین سناریوی حمله جبری روی  سامانه‌های رمزنگاری متقارن و نامتقارن، به مطالعه‌ی تعدادی از روش‌های خاص می‌پردازیم. به‌ویژه حمله‌های 
\lr{XL}, 
\lr{XSL}
و
\lr{MutantXL}
را مورد مطالعه قرار می‌دهیم که بر پایه‌ی روش‌های خطی‌سازی دستگاه‌های معادلات چندجمله‌ای چندمتغیره هستند. در ادامه حمله‌های جبری مبتنی بر پایه‌ گروبنر و پایه‌های مرزی و نوع دیگری از حمله‌های جبری که بر مبنای روش‌های برنامه‌ریزی با عدد صحیح و مسئله صدق‌پذیری هستند را مطالعه می‌کنیم. 
\vskip 2cm
\textbf{\rl{کلمات کلیدی}:}\\
\textit{سامانه‌های رمزنگاری، حمله‌های جبری، حل دستگاه معادلات چندجمله‌ای، بازخطی سازی، پایه‌های گروبنر، پایه‌های مرزی، برنامه‌ریزی با عدد صحیح، مسئله صدق‌پذیری}
%بازگردانی تنظیم فاصله به حالت قبل:
\baselineskip=0.75cm