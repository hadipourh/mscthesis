% مراجع همگی در یک فایل bibtex با پسوند .bib  وجود دارد، که می بایست در پوشه اصلی گزارش قرار داده شود.
\addcontentsline{toc}{chapter}{کتاب‌نامه}
%%سبک plain-fa: این سبک متناظر با plain.bst می‌باشد. مراجع بر اساس نام‌خانوادگی نویسندگان، به ترتیب صعودی مرتب می‌شوند. همچنین ابتدا مراجع فارسی و سپس مراجع انگلیسی خواهند آمد. 
%%سبک acm-fa: این سبک متناظر با acm.bst می‌باشد. شبیه plain-fa.bst است. قالب مراجع کمی متفاوت است. اسامی نویسندگان انگلیسی با حروف بزرگ انگلیسی نمایش داده می‌شوند. 
%%سبک unsrt-fa: این سبک متناظر با unsrt.bst می‌باشد. مراجع به ترتیب ارجاع در متن ظاهر می‌شوند. 
%%سبک chicago-fa:  این سبک متناظر با chicago.bst می‌باشد. نیاز به بستهٔ natbib دارد. (مراجع مرتب می‌شوند) 
\bibliographystyle{plain-fa}%{plain-fa, acm-fa, unsrt-fa, chicago-fa}
% وارد کردن مراجع، به عنوان آرگومان ورودی کافی است که نام فایل با پسوند .bib را بدهید. (چون فایل .bib ما در مسیر اصلی قرار ندارد، مسیر را نیز در زیر وارد کرده‌ایم.)
\bibliography{./References/references}




