\addcontentsline{toc}{chapter}{پیشگفتار}
\chapter*{ \textbf{\LARGE{پیشگفتار}}}
\renewcommand{\baselinestretch}{1.4}
\begin{flushleft}
{
 \small{
 الف  \ لام \  میم \  است  \ آغاز  \ کار\\
 که  رمزیست \  از سوی پروردگار\\
%« سوره بقره، آیه ۱»
«امید مجد»
}}
\end{flushleft}
محور اصلی مباحث این پایان‌نامه بر اساس مقاله‌
\begin{latin}
\noindent
Kreuzer, Martin. {\em Algebraic attacks galore!.} Groups–Complexity–Cryptology 1, no. \textbf{2} (2009), 231-259.
\end{latin}
\vspace{-4mm}
\noindent است. 

علم رمزشناسی
%\LTRfootnote{Cryptology}
دارای دو وجه است، یک وجه آن شامل طرّاحی سامانه‌ها و پروتکل‌های رمزنگاری و وجه دیگر آن علم یا هنر تحلیل رمز
%\LTRfootnote{Cryptanalysis}
یا رمزشکنی است، که در آن به مطالعه روش‌هایی برای  شکستن سامانه‌ها و پروتکل‌های رمزنگاری  می‌پردازند.  تمرکز ما در این پایان‌نامه بر روی وجه دوّم است. منظور از شکستن یک سامانه رمزنگاری در ادبیات رمزنگاری، به‌دست آوردن کلید یا متن اصلی با تلاشی کمتر از جست‌و‌جوی فراگیر فضای کلید است، و  به هر تلاشی برای شکستن یک سامانه حمله اتلاق می‌شود. 

از آنجایی‌که هر نگاشت رمزنگاری بین فضاهای برداری با بعد متناهی، روی میدان‌ متناهی را می‌توان به صورت یک نگاشت چند‌جمله‌ای نوشت، بیان مسئله‌  حمله به یک سامانه رمز به صورت مسئله‌  حل دستگاه معادلات  چندجمله‌ای امری دور از انتظار نیست و به چنین رویکردی برای حمله به یک سامانه رمزنگاری 
\textit{حمله  جبری }
گفته می‌شود.

حمله‌های جبری جزء حمله‌های شناخته شده دنیای رمزنگاری است و  تا کنون تحقیقات زیادی  در این زمینه صورت گرفته  است، به‌طوری  که حتی پایه‌گذاران رمزنگاری مدرن نیز ایده‌ اصلی حمله‌های جبری را می‌شناختند. برای مثال شانون در مقاله‌  معروف خود 
\cite{shannon1949communication}
که در سال ۱۹۴۹ منتشر شد جمله‌ای با این مضمون دارد، که شکستن یک سامانه‌ رمزنگاری خوب، نیازمند حداقل همان میزان تلاش برای حل یک دستگاه معادلات پیچیده با تعداد مجهولات فراوان است. با وجود مؤفقیت‌های به‌دست آمده، این حوزه هنوز مسائل باز و حل‌نشده‌ی فراوانی دارد. ایده‌ اصلی حمله‌های جبری را می‌توان در سه گام زیر خلاصه کرد
\begin{enumerate}
	\item
	به‌دست آوردن روابط چند‌جمله‌ای بین متغیر‌های به‌کار رفته در یک سامانه‌ رمزنگاری و تبدیل نگاشت‌ رمزنگاری و رمزگشایی به نگاشت‌های چند‌جمله‌ای
	\item
	جایگذاری مقادیر معلوم در روابط به‌دست آمده و تشکیل یک دستگاه معادلات چندجمله‌ای روی یک میدان متناهی
	\item
	حلّ دستگاه معادلات به‌دست آمده
\end{enumerate}


در این پایان‌نامه سعی بر این بوده، تا گردایه‌ای از حمله‌های جبری را مورد مطالعه و ارزیابی قرار دهیم. در این راستا در فصل اول یک آشنایی مختصر با رمزنگاری خواهیم داشت، سپس در فصل ۲ پایه گروبنر و پایه مرزی را معرفی می‌کنیم که از ابزارهای ما در حمله به سامانه‌های رمزنگاری هستند.

در فصل ۳ بعد از این که ثابت کردیم هر نگاشت رمزنگاری را می‌توان به یک نگاشت چندجمله‌ای تبدیل کرد، نحوه‌ی انجام این‌کار را نیز نشان می‌دهیم. در این رابطه الگوریتم بوخبرگر مولر 
\LTRfootnote{Buchberger-M{\"o}ller}
را که از آن می‌توان برای یافتن روابط چندجمله‌ای بین بیت‌های کلید و متن رمز شده، یا روابط چندجمله‌ای بین بیت‌های متن اصلی و متن رمزشده استفاده کرد، معرفی می‌کنیم. 
در ادامه این فصل سناریو‌های عمومی حمله جبری به سامانه‌های رمزنگاری متقارن و نامتقارن را با ذکر  مثال‌هایی از جبری‌سازی الگوریتم‌های رمزنگاری واقعی،  بررسی خواهیم کرد. 

پس از تبدیل سامانه‌ رمزنگاری به دستگاه معادلات  چند‌جمله‌ای نوبت به بررسی الگوریتم‌های حل‌ این دستگاه‌ها می‌رسد که در فصل ۴ به آن پرداخته‌ایم. این فصل را  با روش‌های مبنی بر خطی‌سازی آغاز می‌کنیم و حمله‌ی 
\lr{XL}
را معرفی خواهیم کرد. این الگوریتم که از روش  خطی‌سازی برای تحویل دستگاه چند‌جمله‌ای به یک دستگاه خطی استفاده می‌کند، اولین بار توسط شمیر
\LTRfootnote{A. Shamir}
و 
کیپنیس
\LTRfootnote{A. Kipnis}
در 
\cite{kipnis1999cryptanalysis}
معرفی شد. بر خلاف برخی از امید‌های اولیه پیچیدگی محاسباتی این روش زیرنمایی نبود و حافظه‌ زیادی مصرف می‌کرد. پس از روشن شدن نقایص 
\lr{XL}،
یکی از اولین پیشنهادها برای بهبود  آن، الگوریتم 
\lr{XSL}
بود که توسط کورتوا
\LTRfootnote{Courtois}
و پیپشیک
\LTRfootnote{Pieprzyk}
در 
\cite{courtois2002cryptanalysis}
ارائه شد و موضوع بحث بعدی ما در این فصل است. اگر چه این روش با بهره گیری از ویژگی کم‌پشتی و ساختار خاص دستگاه‌هایی که از برخی سامانه‌های رمزنگاری به‌دست می‌آیند، در بهبود روش 
\lr{XL}
موفق بوده است،  ولی تنها برای دسته‌ی خاصی از سامانه‌ها موسوم به سامانه‌های 
\lr{XSL}
مناسب است و یک روش کلی برای حل تمام دستگاه‌ها تلقی نمی‌شود. علاوه بر این بحث‌های زیادی بر سر کارایی این روش وجود دارد و عده‌ای از رمزنگارها اعتقادی به کارایی این روش ندارند.  یک پیشنهاد دیگر برای بهبود الگوریتم 
\lr{XL}
الگوریتم 
\lr{MutantXL}
از دینگ
\LTRfootnote{J. Ding}
است که آن را نیز در این فصل معرفی می‌کنیم.

در ادامه‌ی فصل ۴، حمله‌های پایه‌ گروبنر و پایه‌ مرزی را مورد بررسی قرار می‌دهیم. بهترین الگوریتم‌ها برای یافتن پایه‌ گروبنر تا‌کنون الگوریتم‌های 
\lr{F4}
و 
\lr{F5}
از فوجر 
\LTRfootnote{Fauger}
در 
\cite{faugere1999new}
و
\cite{Faugere2002}
هستند که حمله‌های مبنی بر پایه گروبنر در رمزنگاری، از این الگوریتم‌ها برای یافتن پایه گروبنر و حل دستگاه استفاده می‌کنند. محبوبیت این الگوریتم‌‌ها پس از مؤفقیت آن‌ها در شکستن چالش 
\lr{HFE 80}
افزایش یافته است. به‌ویژه پیشرفت‌های‌ اخیر و بهینه‌سازی‌هایی که روی این الگوریتم‌ها صورت گرفته آن‌ها را قدرتمند‌تر ساخته و امنیت چندین سامانه‌ی امضای دیجیتال و رمز دنباله‌ای به واسطه‌ی آن‌ها تحت خطر جدی قرار گرفته است. در ادامه‌ی این فصل الگوریتم بهبود‌یافته‌ پایه مرزی را، به عنوان روشی دیگر برای حل دستگاه‌ها و حمله به سامانه‌های رمزنگاری معرفی می‌کنیم. گرچه تا کنون هیچ حمله‌ی مؤفقی با استفاده از پایه‌های مرزی گزارش نشده ولی به نظر می‌رسد الگوریتم‌های محاسبه‌ پایه مرزی اگر به همان اندازه‌ الگوریتم‌های پایه گروبنر مورد توجه قرار گیرند، می‌توانند روشی چه‌بسا بهتر از روش پایه گروبنر،  برای حمله به سامانه‌های رمزنگاری باشند.  

در ادامه فصل ۴ به معرفی حمله‌های جبری مبنی بر برنامه‌ریزی خطی با عدد صحیح می‌پردازیم. در این بخش الگوریتمی برای تبدیل دستگاه معادلات چندجمله‌ای روی 
$\gf(2)$
به مسئله برنامه‌ریزی خطی با متغیرهای صحیح معرفی می‌کنیم، تا بدین ترتیب بتوانیم از حل‌کننده‌های قدرتمند مسائل برنامه‌ریزی خطی  نیز برای حمله به سامانه‌های رمزنگاری استفاده کنیم. در پایان به یکی از معروف‌ترین مسائل علوم کامپیوتر یعنی مسئله صدق‌پذیری پرداخته‌ایم و بعد از  معرفی الگوریتمی برای تحویل مسئله حل دستگاه  چندجمله‌ای به مسئله صدق‌پذیری، سعی کرده‌ایم  تا قدرت حل‌کننده‌های مسئله صدق‌پذیری را در شکستن سامانه‌های رمزنگاری مورد آزمون قرار دهیم. 
%\begin{flushleft}
%\textbf{حسین هادی‌پور}\;\;\;\;\;\;\;\,\\
%\textbf{دانشگاه تهران، شهریور ۱۳۹۵}
%\end{flushleft} 
